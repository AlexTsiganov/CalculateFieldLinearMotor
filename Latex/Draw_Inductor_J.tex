\documentclass[10pt]{article}
\usepackage{pgfplots}
\usepackage[utf8]{inputenc}
\usepackage[russian]{babel}
\usepackage[letterpaper,landscape,left=10mm, right=10mm, top=10mm, bottom=10mm]{geometry}
\pgfplotsset{compat=1.9}

\begin{document}


\begin{tikzpicture}
\begin{axis}[ title=Рапсределение плотности тока на поверхности индуктора,
xlabel={$x, mm$},
ylabel={$y, mm$},
height=0.35\paperheight,
width=0.9\paperwidth,
view={0}{90},
only marks,
colorbar horizontal,
]
\addplot3[
%surf,
mesh,scatter,
] table[x=x, y=y, z=J0] {/Users/alextsiganov/Documents/University/Projects/CalculateFieldLinearMotor/Data files/Calculate_Inductor_J.dat};

\end{axis}

asdfsdf

\end{tikzpicture}

%%%%%%%%%%%%%%%%%%%%%%%%


\begin{tikzpicture}

\begin{axis}[ title=Рапсределение плотности тока на поверхности индуктора,
%xlabel={$index, i$},
ylabel={$J, mm$},
height=0.6\paperheight,
width=0.8\paperwidth,
xmin=-5,
xmax=2200,
axis x line=bottom,
axis y line=left,
%colorbar,
%colorbar horizontal,
%colorbar style={
%at={(0.5,1.03)},anchor=south,
%xticklabel pos=upper
%},
%axis background/.style={
%shade,top color=white,bottom color=white},
]
%\draw (axis cs:100,\pgfkeysvalueof{/pgfplots/ymin}) -- (axis cs:100,\pgfkeysvalueof{/pgfplots/ymax});


\addplot[
%no markers
%mesh,
%ultra thick,
%mark size = 10mm,
red,
%line join=bevel,
] table[x=n, y=J0] {/Users/alextsiganov/Documents/University/Projects/CalculateFieldLinearMotor/Data files/Calculate_Inductor_J.dat};

\addplot[
%no markers
%mesh,
%ultra thick,
%mark size = 10mm,
blue,
%line join=bevel,
] table[x=n, y=J] {/Users/alextsiganov/Documents/University/Projects/CalculateFieldLinearMotor/Data files/Calculate_Inductor_J.dat};

\end{axis}
\end{tikzpicture}




\end{document}