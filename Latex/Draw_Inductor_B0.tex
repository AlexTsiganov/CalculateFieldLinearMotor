\documentclass[10pt]{article}
\usepackage[usenames]{color} %used for font color
\usepackage{pgfplots}
\usepackage{comment}
\usepackage{pgffor}
\usepackage[utf8]{inputenc}
\usepackage[russian]{babel}
\usepackage[letterpaper, total={130mm,297mm}, landscape,left=2mm, right=2mm, top=2mm, bottom=2mm]{geometry}

\begin{document}

Текст
\begin{tikzpicture}
\begin{axis}[ title=Рапределение значения B0 по границе индуктора (правой части матрицы), xlabel={Номерация точек с координаты (0,0)}, ylabel={B0},
height=80mm,
width=260mm,  xticklabel style={/pgf/number format/1000 sep=},
]

\addplot[
mark size = 0.001mm,
red
] table[x=n, y=z] {/Users/alextsiganov/Documents/University/Projects/CalculateFieldLinearMotor/Data files/Calculate_Inductor_B0.dat};

\end{axis}

asdfsdf

\end{tikzpicture}

%%%%%%%%%%%%%%%%%%%%%%%%

\begin{tikzpicture}
\begin{axis}[ title=Рапределение значения B0 по границе индуктора (правой части матрицы), xlabel={}, ylabel={B0},
height=80mm,
width=260mm,  xticklabel style={/pgf/number format/1000 sep=},
axis x line=center,
axis y line=left
]

\addplot[
mark size = 0.001mm,
red
] table[x=n, y=z] {/Users/alextsiganov/Documents/University/Projects/CalculateFieldLinearMotor/Data files/Calculate_Inductor_B0.data};

\end{axis}

asdfsdf

\end{tikzpicture}

%%%%%%%%%%%%%%%%%%%%%%%%

\begin{tikzpicture}
\begin{axis}[ title=Рапределение значения B0 по границе индуктора (правой части матрицы), xlabel={}, ylabel={B0},
height=140mm,
width=260mm,
view={0}{90},
only marks
]

\addplot3[
%surf,
mesh,scatter,
] table[x=x, y=y, z=z] {/Users/alextsiganov/Documents/University/Projects/CalculateFieldLinearMotor/Data files/Calculate_Inductor_B0.data};

\end{axis}

asdfsdf

\end{tikzpicture}

\end{document}